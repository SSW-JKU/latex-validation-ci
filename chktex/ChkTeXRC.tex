





\subsubsection{The \rsrc\ file format}

The chktexrc file is essentially a bunch of variable assignments.
There are two types of variables, those that take single items and
those that take lists.

In turn, there are two types of lists, case-sensitive and case-insensitive.
Case-sensitive lists are delimited by \verb@{@ and \verb@}@
while case-insensitive are delimited by \verb@[@ and \verb@]@.
Only some variables support case insensitive lists, since in many
cases it doesn't make sense and would be unnecessarily slow.  Those
variables that \emph{do} support case-insensitive lists will be marked as
such throughout the file.

Variables can be set with or without an equals sign.  If included, the
\verb@=@ causes the variable to be overwritten.  This is the only thing
that makes sense for variables taking a single item and so we always
include it in that case.  For list variables, omitting the equals
sign will cause the items in the list to be appended instead of
overwriting the entire list.

Below are all the ways in which a variable can be set.  Note that lists
can span lines, though this is not shown here for brevity.


\begin{verbatim}
VariableName = item
# Overwrites
VariableName = { Item1 Item2 ... }
VariableName = [ item1 item2 ... ]
VariableName = { Item1 Item2 ... } [ item item ... ]
VariableName = [ item1 item2 ... ] { Item Item ... }
# Appends
VariableName { Item3 Item4 ... }
VariableName [ item3 item4 ... ]
VariableName { Item3 Item4 ... } [ item item ... ]
VariableName [ item3 item4 ... ] { Item Item ... }
\end{verbatim}


Comments begin with \verb@#@, and continue for the end of the line.
Blank lines plus leading and trailing spaces are ignored.
Items are separated by spaces.
Newlines are considered spaces, but can't be escaped.
You can use double quotes \verb@"@ to surround an item with spaces, or you can
escape spaces as described later.

Detection of tokens like \verb@}@ are somewhat context sensitive---they
have to be preceded by a space (or newline).  This allows them to be
part of an item without escaping.  Since some variables require such
characters, this generally makes life easier.

To include characters that might otherwise interfere, escape
sequences are provided.  They are similar to those in C, but use
\verb@!@ instead of \verb@\@ for obvious reasons.
The entire list is below.

\vspace{0.5\baselineskip}
\begin{tabular}{p{0.2\textwidth}p{0.7\textwidth}}
\bf Sequence   & \bf Resulting character \\
\verb@! @      &  Space \\
\verb@!"@      &  \verb@"@ \\
\verb@!#@      &  \verb@#@ \\
\verb@!!@      &  \verb@!@ \\
\verb@!{@      &  \verb@{@ \\
\verb@!}@      &  \verb@}@ \\
\verb@![@      &  \verb@[@ \\
\verb@!]@      &  \verb@]@ \\
\verb@!=@      &  \verb@=@ \\
\verb@!b@      &  Backspace \\
\verb@!n@      &  New line \\
\verb@!r@      &  Carriage return \\
\verb@!t@      &  Tab \\
\verb@!f@      &  Form feed \\
\verb@!xNN@    &  NN must be a hexadecimal number (00 - ff),
both characters must be included. \\
\verb@!dNNN@   &  NNN must be a decimal number (000 - 255), all
three characters must be included.
Unspecified results if NNN > 377. \\
\verb@!NNN@    &  NNN must be a octal number (000 - 377), all
three characters must be included.
Unspecified results if NNN > 377.
\end{tabular}

\subsubsection{Settings in the \rsrc\ file}

All available settings follow.


\medskip
\begin{chktexrcsimplevar}{QuoteStyle}

The type of quote-style you are using.  There are currently two
styles:

Traditional:
\begin{errexam}
"An example," he said, "would be great."
\end{errexam}

Logical:
\begin{errexam}
"An example", he said, "would be great".
\end{errexam}

\chktexrcdefault\begin{verbatim}
QuoteStyle = Logical
\end{verbatim}
\end{chktexrcsimplevar}


\begin{chktexrcsimplevar}{TabSize}

The width of a tab.  This is used for formatting the error message.
Only positive integers are allowed.

\chktexrcdefault\begin{verbatim}
TabSize = 8
\end{verbatim}
\end{chktexrcsimplevar}


\medskip
\begin{chktexrcsimplevar}{CmdSpaceStyle}

How to treat a command is followed by punctuation.  In all cases the
warnings are also governed by the main warning settings, namely
warnings 12 and 13 about interword and intersentence spacings.
These can be found on page~\pageref{warn:interword}.

If set to Ignore, then it won't print any warnings when punctuation
follows a command.


If CmdSpaceStyle is set to InterWord, then it will print warnings
when interword spacing should (potentially) be used.  For example,
without a command the following will trigger warning 12

\begin{errexam}
\verb@I've seen a UFOs, etc. in my life.@
\end{errexam}

And if set to InterWord, so will

\begin{errexam}
\verb@I've seen a UFOs, \etc. in my life.@
\end{errexam}


If set to InterSentence, then it will print warnings when
intersentence spacing should (potentially) be used.  For example,
without a command the following will trigger warning 13

\begin{errexam}
\verb@I've seen an UFO! Right over there!@
\end{errexam}

And if set to InterSentence, so will

\begin{errexam}
\verb@I've seen an \UFO! Right over there!@
\end{errexam}


Setting CmdSpaceStyle to Both will cause warnings to be printed in
both cases.

\chktexrcdefault\begin{verbatim}
CmdSpaceStyle = Ignore
\end{verbatim}
\end{chktexrcsimplevar}


\begin{chktexrclistvar}{CmdLine}

Default command-line options.  For instance, you might like to put
\verb@-v2@ here.

\chktexrcdefault\begin{verbatim}
CmdLine
{
}
\end{verbatim}
\end{chktexrclistvar}


\begin{chktexrclistvar*}{UserWarn}

Arbitrary strings to warn about.  You can put here to help you find
your own little foibles.  See also~\hyperref[rc:UserWarnRegex]{UserWarnRegex}.

These patterns will be searched for throughout the text; regardless
of whether they appear as normal text, commands, in math mode, etc.
They are \emph{not} found in comments.

Suppose you define a special command like this:
\begin{verbatim}
\def\unknown{\large\bf??}
\end{verbatim}
which you use whenever you don't have some information at the time
of writing.  Thus, it makes sense to warn on it, and this is a
convenient way to do so.

\chktexrcdefault\begin{verbatim}
UserWarn
{
    \unknown
    # One should write \chktex or Chk\TeX - never ChkTeX.
    ChkTeX
}
[ # You may put case-insensitive patterns here.
]
\end{verbatim}
\end{chktexrclistvar*}


\begin{chktexrclistvar}{UserWarnRegex}

A more sophisticated version of~\hyperref[rc:UserWarn]{UserWarn} using regular
expressions.  Use of these will be automatically disabled if \chktex{}
was built without regular expression support.  Because \chktex{} can be
with support for either POSIX or PCRE flavors of regular expression,
some of the following will not apply in all cases.  An expression
can be defined only when PCRE is enabled by prepending the
expression with \verb@PCRE:@ and similarly with \verb@POSIX:@.

These patterns will be searched for, no matter whether they appear
as normal text, commands, or arguments.  However, they will \emph{not}
match in verbatim environments (see~\hyperref[rc:VerbEnvir]{VerbEnvir}).

Remember that you have to escape (with a \verb@!@) the characters
\verb@"#!=@, as well as spaces and \verb@{}[]@ if they are
proceeded by a space.

When using PCRE regular expressions, you can use \verb@(?i)@ to make
the expression case insensitive.  See the man pages (man pcresyntax)
or the nicely formatted http://perldoc.perl.org/perlre.html for
documentation on the regular expression syntax.  Note, however, that
some the features of perl regular expression are not available such
as running code (callouts), and replacing.

An initial PCRE-style comment \verb@(?# ... )@ can be used
to change what is displayed, thereby reminding yourself how to fix
the problem.  This works even for POSIX expressions.

\chktexrcdefault\begin{verbatim}
UserWarnRegex
{
    (?!#Always! use! \nmid)\\not! *(\||\\mid)

    # Capitalize section when saying Section 6.
    (?!#-1:Capitalize! before! references)PCRE:\b(chapter|(sub)?section|theorem|lemma|proposition|corollary|appendix)~\\ref
    (?!#1:Capitalize! before! references)POSIX:([^[:alnum:]]|^)(chapter|(sub)?section|theorem|lemma|proposition|corollary|appendix)~\\ref

    # Spell it introduction
    # PCRE:(?i)\bintro\b(?!#Spell! it! out.! This! comment! is! not! used.)
    # POSIX:([^[:alnum:]]|^)intro([^[:alnum:]]|$)

    # Pretty tables--see http://texdoc.net/texmf-dist/doc/latex/booktabs/booktabs.pdf
    (?!#-2:Use! \toprule,! \midrule,! or! \bottomrule! from! booktabs)\\hline
    # This relies on it being on a single line, and not having anything
    # else on that line.  With PCRE we could match balanced [] and {},
    # but I wonder if it's worth the complexity...
    (?!#-2:Vertical! rules! in! tables! are! ugly)\\begin\{(array|tabularx?\*?)\}(\[.*\])?\{.*\|.*\}

    (?!#-3:Optional! arguments! []! inside! optional! arguments! []! must! be! enclosed! in! {})PCRE:\[(?!![^\]\[{}]*{(?!![^\]\[{}]*}))[^\]]*\[

}
\end{verbatim}
\end{chktexrclistvar}


\begin{chktexrclistvar}{TeXInputs}

A list of paths where \chktex{} should look for files it \verb@\input@s.
By default, the current directory is searched (not recursively, use
\verb@//@ for that, see below).

A \verb@//@ postfix is supported:
if you append a double path-separator we'll recursively search that
directory's directories.  MS-DOS users must append \verb\\
instead, e.g. \verb@C:\EMTEX\\@.  In order to search an
entire directory tree, you must use \emph{three} slashes,
e.g. \verb@c:\\\@ or \verb@///@.

\chktexrcdefault\begin{verbatim}
TeXInputs
{
}
\end{verbatim}
\end{chktexrclistvar}


\begin{chktexrclistvar}{OutFormat}

Output formats which can be selected from the command-line.  The
\verb@-v@ option simply indexes into this list.  By default, entry number
\emph{two} in this list is chosen (counting from 0), and \verb@-v@
without any parameter selects entry number \emph{three}.

For explanation of the \verb@%@ format specifiers see the
documentation of the \verb@--format@ command-line argument
\begin{comment}
in the manual.
\end{comment}
on page~\pageref{clarg:format}.

Recall that to use \verb@!@ is the escape character, not \verb@\@.

\chktexrcdefault\begin{verbatim}
OutFormat
{
    # -v0; silent mode
    %f%b%l%b%c%b%n%b%m!n
    # -v1; normal mode
    "%k %n in %f line %l: %m!n%r%s%t!n%u!n"
    # -v2; fancy mode
    "%k %n in %f line %l: %m!n%r%i%s%I%t!n!n"
    # -v3; lacheck mode
    "!"%f!", line %l: %m!n"
    # -v4; verbose lacheck mode
    "!"%f!", line %l: %m!n%r%s%t!n%u!n"
    # -v5; no line number, ease auto-test
    "%k %n in %f: %m!n%r%s%t!n%u!n"
    # -v6; emacs compilation mode
    "!"%f!", line %l.%c:(#%n) %m!n"
}
\end{verbatim}
\end{chktexrclistvar}


\begin{chktexrclistvar*}{Silent}

Commands which should not trigger a warning if terminated by a space.
This warning will not trigger in math mode.

You can also specify regular expressions in the \verb@[]@
section in case you have many custom macros that can be safely
terminated with a space.

\chktexrcdefault\begin{verbatim}
Silent
{
    \rm \em \bf \it \sl \sf \sc \tt \selectfont
    \rmfamily \sffamily \ttfamily \mdseries \bfseries \itshape
    \slshape \scshape \relax
    \vskip \pagebreak \nopagebreak

    \textrm \textem \textbf \textit \textsl \textsf \textsc \texttt

    \clearpage \ddots \dotfill \flushbottom \fussy \indent \linebreak
    \onecolumn \pagebreak \pushtabs \poptabs \scriptsize \sloppy
    \twocolumn \vdots
    \today \kill \newline \thicklines \thinlines

    \columnsep \space \item \tiny \footnotesize \small \normalsize
    \normal \large \Large \LARGE \huge \Huge \printindex

    \newpage \listoffigures \listoftables \tableofcontents
    \maketitle \makeindex

    \hline \hrule \vrule

    \centering

    \noindent \expandafter

    \makeatletter \makeatother

    \columnseprule

    \textwidth \textheight \hsize \vsize

    \if \fi \else

    \csname \endcsname

    \z@ \p@ \@warning \typeout

    \dots \ldots \input \endinput \nextline \leavevmode \cdots
    \appendix \listfiles \and \quad \bigskip \medskip \smallskip
    \hskip \vfill \vfil \hfill \hfil \topmargin \oddsidemargin
    \frenchspacing \nonfrenchspacing
    \begingroup \endgroup \par

    \vrefwarning \upshape \headheight \headsep \hoffset \voffset
    \cdot \qquad \left \right \qedhere \xspace

    \addlinespace \cr \fill \frontmatter
    \toprule \midrule \bottomrule
}[
    # Here you can put regular expressions to match Silent macros.  It
    # was designed for use with many custom macros sharing a common
    # prefix, but can of course be used for other things.

    # Support ConTeXt to at least some extent
    \\start.* \\stop.*
]
\end{verbatim}
\end{chktexrclistvar*}


\begin{chktexrclistvar}{HyphDash}

The number of dashes allowed between two alphabetic characters.
Use 0 to always return an error.  For example:
\begin{errexam}
foo-bar \\
Use of two--dashes is not usually allowed in English. \\
like this---see?
\end{errexam}

For English, this will often be a single dash (hyphen).  If you like
m-dashes with no spaces between them and the surrounding words, then
it should include 3 as well.  There \emph{are} cases when an n-dash
is valid between two alphabetic characters.  See~\hyperref[rc:DashExcpt]{DashExcpt}.

\chktexrcdefault\begin{verbatim}
HyphDash { 1 3 }
\end{verbatim}
\end{chktexrclistvar}


\begin{chktexrclistvar}{NumDash}

The number of dashes allowed between two numeric characters.
Use 0 to always return an error.  This does \emph{not} apply in
math mode.  For example:
\begin{errexam}
123--456 is a range \\
$12-4$ \% okay because it's in math mode
\end{errexam}

For English, this should be 2 because an n-dash is used to indicate
a range of numbers and subtraction should be in math mode where this
does not apply.

\chktexrcdefault\begin{verbatim}
NumDash  { 2 }
\end{verbatim}
\end{chktexrclistvar}


\begin{chktexrclistvar}{WordDash}

The number of dashes allowed between two space characters.
Use 0 to always return an error.  For example:
\begin{errexam}
not like - this,  \\
or like -- this.  \\
like this --- see?
\end{errexam}

\chktexrcdefault\begin{verbatim}
WordDash { 3 }
\end{verbatim}
\end{chktexrclistvar}


\begin{chktexrclistvar}{DashExcpt}

Exceptions to the dash rules above.  For example, an n-dash
\verb@--@ between words is usually wrong, but in some cases it is correct,
such as when naming a theorem.  The Birch--Swinnerton-Dyer
conjecture is one example where the difference matters.  You can
tell that Birch is one person and Swinnerton-Dyer is another based
on the dashes used.

Adding line suppressions for these is possible, but can quickly
become tedious if a certain theorem is referenced often.  For this
reason exceptions can be specified here.  They are case-sensitive.

Unfortunately, there are no warnings if the dashes are surrounded by differing
types of characters.  For example:
\begin{errexam}
like this ---see? (space and alphabet)  \\
a--123            (number and alphabet) \\
a.--b.            (other character, namely \verb@.@)
\end{errexam}
Similarly, no warnings are issued if the hyphenation is correct,
according to the other rules, for example:
\begin{errexam}
Birch-Swinnerton-Dyer
\end{errexam}

\chktexrcdefault\begin{verbatim}
DashExcpt
{
    Birch--Swinnerton-Dyer
}
\end{verbatim}
\end{chktexrclistvar}


\begin{chktexrclistvar}{WipeArg}

Commands whose arguments aren't \LaTeX{} code, and thus should be
ignored.

After the command, you may place arguments (separated from the
command with a colon) that should be wiped.  Use \verb@[]@ for optional
arguments, \verb@{}@ for required ones, and \verb@*@ if the command supports a
star variant.  Some commands (e.g.\@ \verb@\cmidrule@) use \verb@()@ to
delimit an optional argument and so this syntax is supported as well.

For instance, if you would like to wipe the \verb@\newcommand@ command,
you would declare it as \verb@\newcommand:*[][]{}@ since it has a
star variant, two optional arguments, and one required argument.

These commands may be ``evaluated'' before they're wiped, so you will
typically list file handling commands and similar here.

\chktexrcdefault\begin{verbatim}
WipeArg
{
    \label:{} \ref:{} \eqref:{} \vref:{} \pageref:{} \index:[]{}
    \cite:[][]{} \nocite:{}
    \input:{} \verbatiminput:[]{} \listinginput:[]{}{}
    \graphicspath:{}
    \verbatimtabinput:[]{} \include:{} \includeonly:{}
    \bibitem:[]{}
    \cline:{} \cmidrule:[](){}
    \href:{}{}
    # Cleveref -- there are many others that could be here as well...
    \cref:*{} \cpageref:*{} \crefrange:*{}{} \cpagerefrange:*{}{}
    \Cref:*{} \Cpageref:*{} \Crefrange:*{}{} \Cpagerefrange:*{}{}
    # natbib
    \citet:*[][]{} \citep:*[][]{} \citealt:*{} \citealp:*[]{} \citeauthor:*{}
    \Citet:*[][]{} \Citep:*[][]{} \Citealt:*{} \Citealp:*[]{} \Citeauthor:{}
    \citetext:{} \citeyear:*{} \citeyearpar:{}
    # biblatex - not including special commands
    \autocite:*[][]{} \autocites:*[][]{} \Autocite:*[][]{} \Autocites:*[][]{}
    \parencite:*[][]{} \parencites:*[][]{} \Parencite:*[][]{} \Parencites:*[][]{}
    \footcite:*{} \footcites:*[][]{} \Footcite:*[][]{} \Footcites:*[][]{}
    \textcite:*{} \textcites:*[][]{} \Textcite:*[][]{} \Textcites:*[][]{}
    \citeauthor:*{} \citeauthors:*[][]{} \Citeauthor:*[][]{} \Citeauthors:*[][]{}
    \citeyear:*{} \citeyears:*[][]{} \Citeyear:*[][]{} \Citeyears:*[][]{}
    \citetitle:*{} \citetitles:*[][]{} \Citetitle:*[][]{} \Citetitles:*[][]{}
    # tipa which uses "
    \textipa:{}
    # LuaTeX
    \directlua:{} \luaescapestring:{}
}
\end{verbatim}
\end{chktexrclistvar}


\begin{chktexrclistvar}{MathEnvir}

Environments which typeset their contents as mathematics.
This turns on/off some warnings.

A \verb@*@ variant is automatically added for each keyword.

\chktexrcdefault\begin{verbatim}
MathEnvir
{
    displaymath math eqnarray array equation
    align alignat gather flalign multline
    dmath dgroup darray
}
\end{verbatim}
\end{chktexrclistvar}


\begin{chktexrclistvar}{TextEnvir}

Environments which typeset their contents as text, for use inside
mathematics.  This turns on/off some warnings.

\chktexrcdefault\begin{verbatim}
TextEnvir
{
    dsuspend
}
\end{verbatim}
\end{chktexrclistvar}


\begin{chktexrclistvar}{MathCmd}

Commands whose argument will be typeset as mathematics.
The commands are assumed to have one mandatory argument which is in
math mode.  This turns on/off some warnings.

\chktexrcdefault\begin{verbatim}
MathCmd
{
    \ensuremath
}
\end{verbatim}
\end{chktexrclistvar}


\begin{chktexrclistvar}{TextCmd}

Commands whose argument will \emph{not} be typeset as
mathematics even if it would otherwise be in math mode.
The commands are assumed to have one mandatory argument which is in
text mode.  This turns on/off some warnings.

\chktexrcdefault\begin{verbatim}
TextCmd
{
    \text \intertext \shortintertext \mbox \condition
}
\end{verbatim}
\end{chktexrclistvar}


\begin{chktexrclistvar}{VerbEnvir}

Environments containing non-\LaTeX{} content of some kind, and
therefore should not trigger any warnings.

A \verb@*@ variant is automatically added for each keyword.

\chktexrcdefault\begin{verbatim}
VerbEnvir
{
    verbatim comment listing verbatimtab rawhtml errexam picture texdraw
    filecontents pgfpicture tikzpicture minted lstlisting IPA
}
\end{verbatim}
\end{chktexrclistvar}


\begin{chktexrclistvar*}{Abbrev}

Abbreviations not automatically handled by \chktex{}.

\chktex{} automagically catches most abbreviations; the ones we need to
list here, are those which are most likely to be followed by a word
with an upper-case letter which is not the beginning of a new
sentence.

The case-insensitive abbreviations are not fully case-insensitive.
Rather, only the first character is case-insensitive, while the
remaining characters are case-sensitive.

To speed up the searching process somewhat, we require that these
end in a \verb@.@ which should not be a problem in practice.

Much of this work (both the abbreviations below, and the regular
expressions necessary to catch the remaining automatically) have
been provided by Russ Bubley, <russ@scs.leeds.ac.uk>.

\chktexrcdefault\begin{verbatim}
Abbrev
{
    # Ordinals
    1st. 2nd. 3rd. 4th.
    # Titles
    Mr. Mrs. Miss. Ms. Dr. Prof. St.
    #
    # Days
    # Mon. Tue. Wed. Thu. Fri. Sat. Sun.
    #
    # Months
    # Jan. Feb. Mar. Apr. May. Jun. Jul. Aug. Sep. Oct. Nov. Dec.
    #
    # Letters
    # Kt. Jr.
    #
    # Corporate
    # Co. Ltd.
    #
    # Addresses
    # Rd. Dr. St. Ave. Cres. Gdns. Sq. Circ. Terr. Pl. Arc. La. Clo. Ho. Est. Gn.
    #
    # Misc.
    # oe. pbab. ps. rsvp. Tx.
}
[
    # The first letter is case-insensitive in the abbrevs in this
    # list. Due to the nature of the checking algorithm used for
    # this, entries consisting of only one character will be
    # silently ignored.
    #
    # Latin
    # cf. "et al." etc. qed. qv. viz.
    #
    # Corporate
    # inc. plc.
    #
    # Misc
    # fax. pcs. qty. tel. misc.
]
\end{verbatim}
\end{chktexrclistvar*}


\begin{chktexrclistvar}{IJAccent}

Commands which add accents above characters.  This means that \verb@\i@ or \verb@\j@
(\verb@\imath@ and \verb@\jmath@ in math mode) should be used instead of \verb@i@ and \verb@j@.

Other accent commands such as \verb@\c@, \verb@\d@, and \verb@\b@, put their accent under
the character, and thus should be used with normal \verb@i@s and \verb@j@s.

\chktexrcdefault\begin{verbatim}
IJAccent
{
    \hat \check \breve \acute \grave \tilde \bar \vec \dot \ddot
    \' \` \^ \" \~ \= \. \u \v \H \t
}
\end{verbatim}
\end{chktexrclistvar}


\begin{chktexrclistvar}{Italic}

Commands which need italic correction when the group is terminated.

\chktexrcdefault\begin{verbatim}
Italic
{
    \it \em \sl \itshape \slshape
}
\end{verbatim}
\end{chktexrclistvar}


\begin{chktexrclistvar}{NonItalic}

Commands which makes the font non-italic.

\chktexrcdefault\begin{verbatim}
NonItalic
{
    \bf \rm \sf \tt \sc
    \upshape
}
\end{verbatim}
\end{chktexrclistvar}


\begin{chktexrclistvar}{ItalCmd}

Commands which put their argument into italic (and thus possibly
needs italic correction in the end).

This is currently empty, since \verb@\textit@, \verb@\textsl@, and \verb@\emph@
automatically add italic correction.

\chktexrcdefault\begin{verbatim}
ItalCmd
{
}
\end{verbatim}
\end{chktexrclistvar}


\begin{chktexrclistvar}{PostLink}

Commands in front of which a page break is highly undesirable.
Thus there should be no space in front of them.

\chktexrcdefault\begin{verbatim}
PostLink
{
    \index \label
}
\end{verbatim}
\end{chktexrclistvar}


\begin{chktexrclistvar}{NotPreSpaced}

Commands that should not have a space in front of them for various
reasons.  Much the same as~\hyperref[rc:PostLink]{PostLink}, but produces a different warning.

\chktexrcdefault\begin{verbatim}
NotPreSpaced
{
    \footnote \footnotemark \/
}
\end{verbatim}
\end{chktexrclistvar}


\begin{chktexrclistvar}{Linker}

Commands that should be prepended with a \verb@~@.  For example
\begin{errexam}
\verb@look in table~\ref{foo}@
\end{errexam}
to avoid the references being split across lines.

\chktexrcdefault\begin{verbatim}
Linker
{
    \ref \vref \pageref \eqref \cite
}
\end{verbatim}
\end{chktexrclistvar}


\begin{chktexrclistvar}{CenterDots}

Commands or characters which should have \verb@\cdots@ in between.
For example, $1+2+3+\cdots+n$.

\chktexrcdefault\begin{verbatim}
CenterDots
{
    = + - \cdot \div & \times \geq \leq < >
}
\end{verbatim}
\end{chktexrclistvar}


\begin{chktexrclistvar}{LowDots}

Commands or characters which should have \verb@\ldots@ in between.
For example, $1,2,3,\ldots,n$.

\chktexrcdefault\begin{verbatim}
LowDots
{
    . , ;
}
\end{verbatim}
\end{chktexrclistvar}


\begin{chktexrclistvar}{MathRoman}

Words that should appear in roman (upright) in math mode.  There are
certain aliases for mathematical operators (like sin or cos) that
appear in roman rather than the usual italic (slanted) font.

These entries do not need a leading slash since the mistake is often
to \emph{not} include the leading slash.

\chktexrcdefault\begin{verbatim}
MathRoman
{
    log lg ln lim limsup liminf sin arcsin sinh cos arccos cosh tan
    arctan tanh cot coth sec csc max min sup inf arg ker dim hom det
    exp Pr gcd deg bmod pmod mod
}
\end{verbatim}
\end{chktexrclistvar}


\begin{chktexrclistvar}{Primitives}

Commands that are used in \TeX{} but have become unnecessary in
\LaTeX{}, as there are \LaTeX{} commands which do the same.  Purists
should thus avoid these in their code.

\chktexrcdefault\begin{verbatim}
Primitives
{
   \above \advance \catcode \chardef \closein \closeout \copy \count
   \countdef \cr \crcr \csname \delcode \dimendef \dimen \divide
   \expandafter \font \hskip \vskip \openout
}
\end{verbatim}
\end{chktexrclistvar}


\begin{chktexrclistvar}{NoCharNext}

Commands and a set of characters that should \emph{not} follow them.  For
example, in math mode, \verb@\left@ should be followed by a delimiter
which is to change size.  Therefore, it should not be followed by the
end of math mode \verb@$@ or a grouping character \verb@{@ or \verb@}@.

The format is \verb@\command:characters@.

\chktexrcdefault\begin{verbatim}
NoCharNext
{
   \left:{}$ \right:{}$
}
\end{verbatim}
\end{chktexrclistvar}


\begin{chktexrcsimplevar}{VerbClear}

The character to replace verbatim text with.

The arguments of commands listed in~\hyperref[rc:WipeArg]{WipeArg}, as well as
\verb@\verb+...+@ commands, are replaced with an innocuous character
to prevent that data from inadvertently producing a warning.

This should not contain an alphabetic character (in case the user
writes (\verb@\foo\verb+bar+@), neither should it contain be one of
\LaTeX{}'s reserved characters (\verb@#$%&~_^\{}@), or any parenthesis
character (\verb@()[]{}@).  If possible, don't use a punctuation
character or any spacing characters either.  All of these characters
have warnings associated with them and thus could cause spurious
warnings to appear.

The asterisk is also unsuitable, as some commands behave in another
way if they are appended with an asterisk.  Which more or less
leaves us with the pipe.

Please note that this may also be a \verb@string@, which will be
repeated until the proper length is reached.

\chktexrcdefault\begin{verbatim}
VerbClear = "|"
\end{verbatim}
\end{chktexrcsimplevar}


